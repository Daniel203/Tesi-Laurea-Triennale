\chapter{Conclusione}
\label{cha:conclusione}
All'interno di questa tesi abbiamo analizzato la tecnologia
della blockchain, studiandone le basi teoriche fino ad arrivare ad
un'applicazione pratica. 

Abbiamo visto come le caratteristiche di decentralizzazione e integrità dei dati
possano cambiare il modo in cui i nostri dati vengono salvati, creando delle
nuove opportunità per lo sviluppo di applicazioni sicure da questo punto di 
vista. Ci siamo imbattuti nello studio delle fondamenta di una blockchain come
la definizione di catena e di blocco, fino ad arrivare a capire come vengono
effettivamente validate le transazioni e i singoli dati. 
Dopo aver capito come funziona una blockchain, abbiamo cominciato ad analizzare
quella di Ethereum nello specifico andando a vedere come funziona la Ethereum
Virtual Machine e cosa sono gli smart contract. Abbiamo studiato come si scrive
un programma che viene salvato all'interno della blockchain e ne abbiamo analizzato
i punti di forza e gli svantaggi rispetto a una normale applicazione. 

Non ci siamo solo fermati a studiare in modo teorico la blockchain, abbiamo
anche provato ad implementare direttamente uno smart contract all'interno di 
un contesto aziendale per poter effettivamente toccare con mano questo mondo
potendone vedere le potenzialità, ma anche i limiti. Abbiamo visto che lo
sviluppo non è eccessivamente complesso, tuttavia i costi sono ancora un
problema che allontana molto la possibilità di vedere la blockchain integrata
nei sistemi che usiamo tutti i giorni.

In conclusione, la tecnologia della blockchain è sicuramente uno strumento 
innovativo che porta un nuovo paradigma per quanto riguarda la gestione dei dati
salvati, sicuramente qualcosa di diverso da quello a cui siamo abituati. 
La sua decentralizzazione porta dei notevoli vantaggi per quanto riguarda la
privacy e il distaccamento da un'entità centrale, garantendo anche l'operatività
continua e senza interruzioni. Tuttavia come abbiamo visto, ci sono ancora 
dei limiti che non ci permettono di utilizzare la blockchain quotidianamente,
come l'elevato costo per eseguire qualsiasi operazione e anche a causa della
conoscenza richiesta per poterla utilizzare.
