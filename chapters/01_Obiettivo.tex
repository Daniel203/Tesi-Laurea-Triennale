\section{Obiettivo}
L'azienda in cui ho svolto l'attività di stage universitario opera
principalmente nel settore assicurativo producendo gestionali per
semplificare l'amministrazione degli utenti, delle scadenze delle polizze,
degli avvisi di pagamento, delle rateizzazione...
Un compito importante che deve svolgere il programma venduto è il salvataggio
di documenti che, molto spesso, sono contratti firmati dagli utenti e per
questo motivo è fondamentale garantirne l'integrità e l'autenticità per tutto
il periodo di conservazione.

Durante il periodo di stage mi è stato chiesto di realizzare un sistema che
permettesse di salvare i dati in un posto sicuro che garantisse la qualità dei 
dati salvati e che provasse l'autenticità. L'idea che è stata subito proposta 
è stata quella di utilizzare l'infrastruttura della blockchain in quanto per 
sua natura offre un sistema teoricamente immutabile e per questo perfetto per
il nostro scopo.

L'idea che subito è venuta in mente era quella di salvare tutti i dati del file
all'interno della blockchain, ma subito sono emersi alcuni dubbi e problemi, 
in particolare:
\begin{itemize}
    \item i dati sarebbero stati ridondanti in quanto i documenti venivano già 
        salvati sui database aziendali e/o dei clienti
    \item esporre informazioni private su un sistema online poteva portare a dei
        rischi per la privacy dei dati dei clienti 
    \item lo storage online può avere potenzialmente costi maggiori rispetto ad un 
        sistema in locale
\end{itemize}
% L'idea è stata quella di riuscire in qualche modo a salvare un qualche dato 
% che facesse riferimento al documento, ma senza caricare il documento intero in
% quanto sarebbe ridondante essendo il file già salvato sui server aziendali 
% e/o dei cilenti. 
