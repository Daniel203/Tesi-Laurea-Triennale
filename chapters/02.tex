\section{Blockchain}

\subsection{Cos'è?}
Una blockchain è una struttura pubblica, condivisa e immutabile che permette di
salvare dati all'interno di blocchi. I blocchi sono collegati tra loro in modo
tale che ogni blocco verifichi il successivo e per questo modo se un nodo della
catena si rompe, allora tutta la catena viene invalidata.

\subsection{Parole chiave}
Si usa parola \textbf{blocco} per indicare un insieme di informazioni che
vengono salvate all'interno della blockchain. Infatti i dati non sono inseriti
singolarmente come in un database tradizionale, ma vengono raggruppati in
blocchi di dimensione predefinita e una volta che viene soddisfatta la
dimensione vengono salvati.

Una \textbf{catena} è l'insieme dei blocchi collegati tra loro. Ogni blocco è 
collegato al blocco precedente tramite un codice hash creando in questo modo
un legame inviolabile, dato che la modifica di un blocco porterebbe al 
cambiamento del codice hash e quindi all'invalidazione di tutti i collegamenti
successivi.

Un \textbf{hash code} è un codice che viene generato da una funzione matematica
che a partire da una stringa di lunghezza variabile, ne genera una di lunghezza
fissa che identifica univocamente la stringa di partenza. Può essere visto come
se fosse l'impronta digitale dell'elemento di partenza in quanto è unica e 
rappresenta in modo diretto l'input iniziale. Il codice hash risulta differente
anche se un solo carattere della stringa di partenza viene modificato, quindi è 
facile intuire che se un blocco cambia hash code, allora sicuramente è stato
mutato.

Ogni computer all'interno della rete è chiamato \textbf{nodo} e tiene in 
memoria una copia della rete completa. Questo permette alla tecnologia di 
essere distribuita e decentralizzata, in quanto non esiste un server centrale.

Le operazioni che vengono svolte all'interno di questa struttura sono chiamate
\textbf{transazioni} e sono salvate all'interno dei blocchi.

Ora che abbiamo definito le parole chiave della blockchain possiamo andare ad
analizzare in dettaglio il funzionamento di questa tecnologia.

\newpage

\subsection{Come funziona?}
In questo capitolo andremo ad analizzare il funzionamento di una blockchain,
spiegando i vari passaggi che la contraddistinguono. Prenderemo come
riferimento la blockchain Ethereum in quanto è quella che verrà utilizzata
nell'esempio pratico, tuttavia la maggior parte dei concetti è applicabile 
a tutte le altre blockchain.

\subsubsection{Transazione}
Una transazione, come detto prima, è un'operazione svolta da un utente
all'interno della blockchain. 

\begin{figure}[H]
    \centering
    \includegraphics[width=0.7\textwidth]{TransactionSchema.png} 
    \caption{Cambio di stato con transazione}
    \label{fig:transactionSchema}
\end{figure}

Le transazioni che modificano lo stato della struttura devono essere notificate
a tutti gli utenti che vi partecipano. Per fare ciò si utilizza un sistema di
notifiche chiamato \textbf{broadcast} che permette di inviare un messaggio a
tutti i nodi della rete, in questo modo ogni nodo può mandare una richiesta di
esecuzione di una transazione a tutti e in seguito un validatore la eseguirà e
propagherà il cambiamento di stato su tutta la rete. \\
Le transazioni una volta eseguite vengono salvate all'interno di un blocco che
verrà poi aggiunto alla catena.

\subsubsection{Blocco}
La blockchain è per certi versi simile a un database, ossia il suo scopo è
salvare dati di vario tipo, ma ciò che li differenzia è il modo in cui i dati
vengono salvati. \\ 
Come detto prima la blockchain è una serie di blocchi che contengono
informazioni, in particolare vengono salvati tre dati:
\begin{itemize}
    \item \textbf{Dati}: sono i dati che devono essere salvati. La tipologia di 
        dati può variare in base al tipo di blockchain utilizzata.
    \item \textbf{Hash del blocco precedente}: è l'hash del blocco precedente
        e serve per collegare i blocchi tra loro. Se il blocco precedente viene
        modificato e il suo hash code  cambia, allora non sarà più uguale a
        quello salvato nel blocco successivo e quindi la catena viene
        invalidata.
    \item \textbf{Hash}: è un codice univoco che identifica il blocco e viene
        calcolato in base ai dati contenuti nel blocco stesso. Per poterlo
        calcolare si una appunto una funzione di hashing che prende in input
        i dati e restituisce l'hash.
\end{itemize}

\begin{figure}[H]
    \centering
    \includegraphics[width=0.9\textwidth]{BlocksSchema.png} 
    \caption{Schema dei blocchi Blockchain}
    \label{fig:chain}
\end{figure}

Dall'immagine precedente \ref{fig:chain} si può notare che ogni blocco
è collegato al blocco prima tramite l'hash del blocco precedente. 
Il primo blocco ha come hash precedente \textit{0000} perchè essendo il blocco
iniziale non esistono blocchi prima di lui.

\begin{figure}[H]
    \centering
    \includegraphics[width=0.9\textwidth]{BlocksSchemaInvalid.png} 
    \caption{Catena invalida}
    \label{fig:invalidChain}
\end{figure}

In questo caso \ref{fig:invalidChain} si può notare che il blocco 2 è stato
modificato e quindi il suo hash code è cambiato. Questo ha portato alla
invalidazione della catena in quanto il blocco 3 non ha più come hash del
blocco precedente quello del blocco 2.

\subsubsection{Ethereum Virtual Machine (EVM)}
Nella blockchain di Ethereum c'è un solo computer che ha uno stato a cui
ogni partecipante della rete deve fare riferimento. Ogni partecipante ne tiene
una copia e ha inoltre la possibilità di richiedere al computer principale di
eseguire delle operazioni. Quando questa richiesta viene avanzata, altri
nodi della rete si prendono in carico l'operazione e la verificano, validano ed
eseguono notificando poi agli altri ciò che è avvenuto.
Ogni operazione porta al cambiamento dello stato della EVM, e quando ciò
avviene, il nuovo stato viene aggiornato su tutti i nodi della rete. \\
Questo modello di \textbf{macchina a stati distribuita} permette l'esistenza
degli smart contract, ossia dei programmi scritti dagli utenti che possono
essere eseguiti all'interno della blockchain. Questa caratteristica differenzia
la blockchain di Ethereum da quella di Bitcoin in quanto quest'ultima non
permette l'esecuzione di programmi, limitando di molto le possibilità di
utilizzo della rete.
\newpage

\begin{figure}[H]
    \centering
    \includegraphics[width=0.9\textwidth]{EVMSchema.png} 
    \caption{Ethereum Virtual Machine}
    \label{fig:EVM}
\end{figure}

Lo schema \ref{fig:EVM} mostra che la EVM ha uno stato globale che è immutabile
ed è quello su cui tutti devono fare riferimento, ma poi ogni singola macchina
ha il suo stato locale che può essere modificato e poi in caso notificato alle
altre macchine.

Il cambio di stato si comporta come una funzione matematica, ossia dato un
input restituisce un output in modo deterministico. In modo più formale 
possiamo descrivere la funzione come segue:
\begin{equation}
    \label{eq:stateChange}
    S_{t+1} = \Upsilon(S_t, T)
\end{equation}

Dato uno stato $S_t$ e una transazione $T$ viene restituito un nuovo stato
$\S_{t+1}$.

\subsubsection{Gas fee}
\textbf{Gas} si riferisce all'unità che misura la quantità di sforzo computazionale
richiesto per eseguire operazioni specifiche sulla rete Ethereum. \\
Poiché ogni transazione di Ethereum richiede risorse computazionali per essere
eseguita, tali risorse devono essere pagate per garantire che Ethereum non sia
vulnerabile allo spam e non si blocchi in loop computazionali infiniti. Il
pagamento per la computazione avviene sotto forma di tassa sul gas. \\
\textbf{Gas fee} è la quantità di gas necessaria per eseguire una transazione
moltiplicata per il prezzo di un singolo gas. Il prezzo del gas è espresso in
\textbf{gwei}, che è un sottomultiplo di un ether (1 gwei = $10^{-9}$ ETH).

\textbf{Come vengono calcolate le gas fee?} \\
È possibile impostare la quantità di gas che si è disposti a pagare quando si
invia una transazione. Offrendo una certa quantità di gas, si fa un'offerta per
includere la transazione nel blocco successivo. Se si offre troppo poco, è meno
probabile che i validatori scelgano la transazione per l'inclusione, il che
significa che la transazione potrebbe essere eseguita in ritardo o non essere
eseguita affatto. Se si offre troppo, c'è il rischio di sprecare un po' di ETH.

Il costo totale in gas che si deve pagare è diviso in due parti: 
\textit{base fee} e \textit{priority fee (tip)}. \\
La \textbf{base fee} è la quantità richiesta dal protocollo che bisogna pagare
per poter ritenere l'operazione valida. La \textbf{priority fee} è la quantità
di gas che si offre in più per far eseguire la transazione prima di altre. \\
Per esempio, se si vuole inviare una transazione che richiede 21.000 gas, la
\textit{base fee} è di 10gwei e si offre 1 gwei come \textit{priority fee},
allora il costo totale sarà: 
\begin{center}
units of gas used * (base fee + priority fee)
\end{center}
\begin{equation}
    \label{eq:gasFee}
    21.000 * (10 + 1) = 0.000021 ETH
\end{equation}

\subsubsection{Ether}
Abbiamo visto che le gas fee vengono pagate in ETH, ma cos'è l'ETH? \\
\textbf{Ether} (o ETH) è la criptovaluta nativa della blockchain di Ethereum, ciò
significa che gli scambi di moneta all'interno di questa struttura avvengono
con questa valuta. \\
Gli Ether vengono creati come ricompensa per l'azione di proposta e validazione
di un nuovo blocco. L'importo emesso dipende dal numero di validatori e dalla
quantità di Ether che hanno depositato nei loro portafogli digitali. \\
Di solito 1/8 del valore viene destinato a chi ha proposto il blocco e il resto
viene suddiviso tra i validatori. \\
Ovviamente, per evitare il deprezzamento dovuto a inflazione, alcuni Ether 
devono anche essere eliminati, per questo motivo quando un utente effettua
una transazione, la \textit{base fee}, il cui valore viene impostato dalla
rete in base alla domanda, viene eliminata.


\subsubsection{Meccanismo del consenso}
Il termine meccanismo di consenso si riferisce all'intera serie di protocolli,
incentivi e idee che consentono a una rete di nodi di concordare lo stato di
una blockchain. \\
Ci sono due tipi di meccanismi di consenso: 
\begin{itemize}
    \item \textbf{Proof of Work (PoW)}
    \item \textbf{Proof of Stake (PoS)}
\end{itemize}

\textbf{Proof of Work} \\
Questo meccanismo viene utilizzato all'interno della blockchain di Bitcoin e
veniva utilizzato anche in quella di Ethereum prima del passaggio a Proof of
Stake. Si basa sul concetto di \textbf{mining}, ossia il lavoro svolto per
aggiungere un blocco valido all'interno della catena3 
\begin{itemize}
    \item \textbf{Nuovi blocchi:} I minatori competono per creare nuovi
        blocchi pieni di transazioni elaborate. Il vincitore condivide il nuovo
        blocco con il resto della rete e guadagna alcune criptovalute appena
        coniate (ad esempio BTC nella blockchain Bitcoin). La gara è vinta dal
        computer che riesce a risolvere più velocemente un puzzle matematico e
        la soluzione di questo puzzle è il lavoro di "proof-of-work". Questo
        produce il collegamento crittografico tra il blocco corrente e quello
        precedente. 
    \item \textbf{Sicurezza:} La rete è sicura grazie al fatto che servono il
        51\% della potenza di calcolo totale della rete per poter alterare la
        catena, una quantità di risorse ed energia talmente grande che rende
        assolutamente nulli i possibili benefici di un attacco.
\end{itemize}

\newpage

\textbf{Proof of Stake} \\
Meccanismo utilizzato in questo momento da Ethereum, in cui i validatori sono
solo nodi che hanno depositato una certa quantità di ETH (in questo momento 32
ETH). 
\begin{itemize}
    \item \textbf{Nuovi blocchi:} I validatori vengono scelti in modo casuale
        basandosi sulla quantità di ETH che hanno depositato. Una volta scelti
        creano un nuovo blocco e lo condividono con il resto della rete,
        ricevendo una ricompensa per il loro lavoro oppure una sanzione
        economica in caso di comportamento scorretto, come ad esempio la
        distruzione di una parte o la totalità degli ETH salvati nel
        portafoglio.
    \item \textbf{Sicurezza:} La rete è sicura perché nel caso un attaccante
        volesse commettere un'azione malevole, metterebbe a rischio gran parte
        dei suoi ETH e quindi molto spesso non è conveniente.
\end{itemize}

\subsubsection{Smart Contract}
Uno smart contract è un programma che gira sulla blockchain di Ethereum. È una
raccolta di codice (le sue funzioni) e di dati (il suo stato) che risiede a un
indirizzo specifico sulla blockchain di Ethereum.
