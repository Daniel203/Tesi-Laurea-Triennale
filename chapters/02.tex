\section{Blockchain}

\subsection{Cos'è la Blockchain?}
Una blockchain è una struttura dati pubblica, condivisa e immutabile.
Questa tecnologia permette di salvare dati all'interno di blocchi che sono 
collegati tra loro in modo tale che ogni blocco verifichi il successivo e per 
questo modo se un nodo della catena si rompe, allora tutta la catena viene
invalidata.

Si usa parola \textbf{blocco} per indicare un insieme di informazioni che
vengono salvate all'interno della blockchain. Infatti i dati non sono inseriti
singolarmente come in un database tradizionale, ma vengono raggruppati in
blocchi di dimensione predefinita e una volta che viene soddisfatta la
dimensione venogono salvati.

Una \textbf{catena} è l'insieme dei blocchi collegati tra loro. Ogni blocco è 
collegato al blocco precedente tramite un codice hash creando in questo modo
un legame inviolabile, dato che la modifica di un blocco porterebbe al 
cambiamento del codice hash e quindi all'invalidazione di tutti i collegamenti
successivi.

Ogni computer all'interno della rete blockchain è chiamato \textbf{nodo} e
si occupa di verificare che i blocchi siano validi e di aggiungerli alla
catena, in questo modo la catena non è presente su un solo computer, ma è 
\textbf{distribuita} su tutti i nodi della rete.

Ora che abbiamo definito le parole chiave della blockchain possiamo andare ad
analizzare in dettaglio il funzionamento di questa tecnologia.
