\section{Blockchain}
    \subsection{Perchè Blockchain?}
    La scelta di salvare i documenti utilizzando questa infrastruttura è dovuta a una caratteristica specifica di questa tecnologia: \textbf{l`immutabilità}.\\
    Infatti per come è realizzata la blockchain, ogni dato salvato non può più essere modificato una volta che è stato "autorizzato" da tutta la community di minatori.
    
    \subsection{Definizione di blockchain}
    Una blockchain è una catena di blocchi contenenti una serie di transazioni, la cui validazione si basa sul principio del consenso. Ogni blocco ha bisogno di essere verificato prima di poter essere caricato nella rete e questa operazione può essere controllata da qualsiasi persona grazie al fatto che i dati sulle transazioni e sui blocchi sono accessibili da tutti e senza nessun tipo di oscuramento tramite tecniche di criptazione. Una volta che un blocco è nella rete sarà rappresentato da un hash-code ottenuto tramite un calcolo sui propri dati uniti all'hash-code del blocco precedente e per questo motivo se un blocco dovesse essere modificato verrebbero invalidati tutti i codici dei blocchi successivi, rendendo palese l'operazione non consentita.
    
    \subsection{Smart Contract}
    Una caratteristica imprortante di alcune blockchain, tra cui la più nota e importante che è Ethereum, è la possibilità di "ospitare" gli smart contract.\\
    Uno smart contract è un programma salvato in blockchain che viene eseguito quando una serie di condizioni si verificano. Si comporta come una macchina a stati, quindi può mutare il proprio stato in base a come è stato programmato, ma non può compiere operazioni più complesse come leggere un file o eseguire chiamate a API esterne. La cosa che rende unica questa tecnologia è il fatto che lo stato del contratto è visibile a tutti, e il codice una volta salvato in blockchain non può più essere modificato in quanto funziona come un normale blocco e la sua modifica invaliderebbe la catena di hash code.\\
    Quindi gli smart contract sono dei programmi salvati in blockchain che funzionano come macchine a stati, e una volta caricati diventano immutabili.\\
    È doveroso precisare che il codice sorgente non è caricato in chiaro, quindi vi è anche un atto di fiducia quando se ne utilizza uno.
    %\todo[size=\small]{Potrei parlare della evm, di come funziona il cambio stato all'interno di un blocco, di come viene gestita la memoria, di come i dati vengono compilati in bytecode...}
    